% Title:  GNU Emacs Survival Card

% Copyright (C) 2000-2012  Free Software Foundation, Inc.

% Author: Wlodek Bzyl <matwb@univ.gda.pl>
% Czech translation: Pavel Jan�k <Pavel@Janik.cz>, March 2001
% Slovak translation: Miroslav Vasko <vasko@debian.cz>, March 2001

% This file is part of GNU Emacs.

% GNU Emacs is free software: you can redistribute it and/or modify
% it under the terms of the GNU General Public License as published by
% the Free Software Foundation, either version 3 of the License, or
% (at your option) any later version.

% GNU Emacs is distributed in the hope that it will be useful,
% but WITHOUT ANY WARRANTY; without even the implied warranty of
% MERCHANTABILITY or FITNESS FOR A PARTICULAR PURPOSE.  See the
% GNU General Public License for more details.

% You should have received a copy of the GNU General Public License
% along with GNU Emacs.  If not, see <http://www.gnu.org/licenses/>.


% See survival.tex.

% Process the file with `csplain' from the `CSTeX' distribution (included
% e.g. in the TeX Live CD).

% User interface is `plain.tex' and macros described below
%
% \title{CARD TITLE}{for version 23}
% \section{NAME}
% optional paragraphs separated with \askip amount of vertical space
% \key{KEY-NAME} description of key or
% \mkey{M-x LONG-LISP-NAME} description of Elisp function
%
% \kbd{ARG} -- argument is typed literally

%**start of header


\def\plainfmtname{plain}
\ifx\fmtname\plainfmtname
\else
  \errmessage{This file requires `plain' format to be typeset correctly}
  \endinput
\fi

% PDF output layout.  0 for A4, 1 for letter (US), a `l' is added for
% a landscape layout.
\input pdflayout.sty
\pdflayout=(0)

% Slovak hyphenation rules applied
\shyph

\def\versionemacs{24}           % version of Emacs this is for
\def\year{2012}                 % latest copyright year

\def\copyrightnotice{\penalty-1\vfill
  \vbox{\smallfont\baselineskip=0.8\baselineskip\raggedcenter
    Copyright \copyright\ \year\ Free Software Foundation, Inc.\break
    Pre GNU Emacs \versionemacs\break
    W{\l}odek Bzyl (matwb@univ.gda.pl)\break
    Do �e�tiny prelo�il Pavel Jan�k (Pavel@Janik.cz)\break
    Do sloven�iny prelo�il Miroslav Va�ko (vasko@debian.cz)

    K�pie tohto dokumentu m��ete vytv�ra� a ��ri�
    za predpokladu, �e bud� obsahova� t�to pozn�mku
    o autorsk�ch pr�vach.\par}}

\hsize 3.2in
\vsize 7.95in
\font\titlefont=csss10 scaled 1200
\font\headingfont=csss10
\font\smallfont=csr6
\font\smallsy=cmsy6
\font\eightrm=csr8
\font\eightbf=csbx8
\font\eightit=csti8
\font\eighttt=cstt8
\font\eightmi=csmi8
\font\eightsy=cmsy8
\font\eightss=cmss8
\textfont0=\eightrm
\textfont1=\eightmi
\textfont2=\eightsy
\def\rm{\eightrm} \rm
\def\bf{\eightbf}
\def\it{\eightit}
\def\tt{\eighttt}
\def\ss{\eightss}
\baselineskip=0.8\baselineskip

\newdimen\intercolumnskip % horizontal space between columns
\intercolumnskip=0.5in

% The TeXbook, p. 257
\let\lr=L \newbox\leftcolumn
\output={\if L\lr
    \global\setbox\leftcolumn\columnbox \global\let\lr=R
  \else
       \doubleformat \global\let\lr=L\fi}
\def\doubleformat{\shipout\vbox{\makeheadline
    \leftline{\box\leftcolumn\hskip\intercolumnskip\columnbox}
    \makefootline}
  \advancepageno}
\def\columnbox{\leftline{\pagebody}}

\def\newcolumn{\vfil\eject}

\def\bye{\par\vfil\supereject
  \if R\lr \null\vfil\eject\fi
  \end}

\outer\def\title#1#2{{\titlefont\centerline{#1}}\vskip 1ex plus 0.5ex
   \centerline{\ss#2}
   \vskip2\baselineskip}

\outer\def\section#1{\filbreak
  \bskip
  \leftline{\headingfont #1}
  \askip}
\def\bskip{\vskip 2.5ex plus 0.25ex }
\def\askip{\vskip 0.75ex plus 0.25ex}

\newdimen\defwidth \defwidth=0.25\hsize
\def\hang{\hangindent\defwidth}

\def\textindent#1{\noindent\llap{\hbox to \defwidth{\tt#1\hfil}}\ignorespaces}
\def\key{\par\hangafter=0\hang\textindent}

\def\mtextindent#1{\noindent\hbox{\tt#1\quad}\ignorespaces}
\def\mkey{\par\hangafter=1\hang\mtextindent}

\def\kbd#{\bgroup\tt \let\next= }

\newdimen\raggedstretch
\newskip\raggedparfill \raggedparfill=0pt plus 1fil
\def\nohyphens
   {\hyphenpenalty10000\exhyphenpenalty10000\pretole